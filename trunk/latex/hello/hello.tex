\documentclass{report}
\usepackage{some,packages}
%% \parindent=0pt % 全文不缩排

%---------------------------------
% 编译 MetaPost 须要的配置
\usepackage{graphicx,mflogo}
\parindent=0pt
\ifx\pdfoutput\undefined
  \DeclareGraphicsRule{*}{eps}{*}{}
\else
  \DeclareGraphicsRule{*}{mps}{*}{}
\fi
%---------------------------------
%# title page 资讯
\title{Aesop Fables}
\author{Aesop\thanks{Thanks to the reader.}
       \and Nobody\thanks{Thanks to nobody.}}
\date{\today}
%---------------------------------
% 目录深度
\setcounter{tocdepth}{2}


%=================================
\begin{document}
\maketitle % 使得可以生成"title page 资讯"

%# 摘要
% 只有 article/report 类别才有 abstract
\begin{abstract}
The tale, the Parable, and the Fable are all common and popular
modes of conveying instruction. Each is distinguished by its own
special characteristics.
\end{abstract}

\tableofcontents %# 生成目录(要放在\maketitle后面)
This is the first experience of \LaTeX.
\chapter{Aesop Fables}
\section{The Ant and the Dove}
 ...


\chapter{换行与缩排}
\noindent % 只对该段落进行缩排
This is my first {\LaTeX} typesetting example.\\[1cm] % 换行 (\\)
This is my first \LaTeX{} typesetting example.\\
This is my first \LaTeX\ typesetting example.\\

中华人民共和国!。
I am Mr. Edward G.J. Lee, G.J. is a abbreviation of my name.\\
I am Mr.\ Edward G.J. Lee, G.J. is a abbreviation of my name.\\
Please see Appendix A. We will be there soon.\linebreak % 也是强迫换行(\linebreak[n])
Please see Appendix A\null. We will be there soon.\\\\\\\\


% report, article, book 的章节标题的格式不太一样
\chapter{章节标题}
This is the first experience of \LaTeX.
\section{The Ant and the Dove}

An ant went to the bank of a river to quench its thirst, and
being carried away by the rush of the stream, was on the
point of drowning.

A Dove sitting on a tree overhanging the water plucked a
leaf and let it fall into the stream close to her. The Ant
climbed onto it and floated in safety to the bank.

\section{The Dog in the Manger}

A dog lay in a manger, and by his growling and snapping
prevented the oxen from eating the hay which had been
placed for them.

\subsection{What?}
``What a selfish Dog!'' said one of them to his companions;
``he cannot eat the hay himself, and yet refuses to allow
\subsection{Why?}
\subsubsection{Why?}
those to eat who can.''


An eagle sat on a lofty rock, watching the movements of a
Hare whom he sought to make his prey.

An archer, who saw the Eagle from a place of concealment,
took an accurate aim and wounded him mortally.


\chapter{加入注解}
\section{脚注(Footnote)}
An ant went to the bank of a river to quench its thirst, and
being carried away by the rush of the stream, was on the
point of drowning.

A Dove\footnote{Pigeon, an emblem of peace.}
sitting on a tree overhanging the water plucked a
leaf and let it fall into the stream close to her. The Ant
climbed onto it and floated in safety to the bank.

\section{边注(Marginal note)}
An ant went to the bank of a river to quench its thirst, and
being carried away by the rush of the stream, was on the
point of drowning.

A Dove\marginpar{Pigeon, an emblem of peace.}
sitting on a tree overhanging the water plucked a
leaf and let it fall into the stream close to her. The Ant
climbed onto it and floated in safety to the bank.
...


\chapter{字型的相关调整}
\section{调整字族、字型系列、字形的指令}
\subsection{The \textsl{Ant} and the \textsl{Dove}}

\itshape
An antwent to the bank of a river to quench its thirst, and
being carried away by the rush of the stream, was on the
point of drowning.
\upshape

A \textsl{Dove} sitting on a tree overhanging the water plucked a
leaf and let it fall into the stream close to her. The \textbf{\textsl{Ant}}
climbed onto it and floated in safety to the bank.

\subsection{The {\it Dog}\/ in the Manger}

A \textbf{\textit{dog}} lay in a manger, and by his growling and snapping
prevented the oxen from eating the hay which had been
placed for them.

``What a selfish Dog!'' said one of them to his companions;
``he cannot eat the hay himself, and yet refuses to allow
those to eat who can.''

\subsection{The \textsc{Eagle} and the Arrow}

An \textsc{eagle} sat on a lofty rock, watching the movements of a
Hare whom he sought to make his prey.

An archer, who saw the \textsc{Eagle} from a place of concealment,
took an accurate aim and wounded him mortally.


\section{相对字型大小的调整}
\begin{small}
took an accurate aim and wounded him mortally.
  本文内容
\end{small}

  ...
\usepackage{type1cm}
不苗地苗羁薯  what lf an
took an accurate aim
% \fontsize{17}{3cm}\seclectfont
  ...


\chapter{原文照列}
The example of \verb|\verb{}| command and \texttt{verbatim} environment.

\section{\textbackslash{}\texttt{verb} command}
% \verb| ... |

When you want to express you home directory, you can \verb|echo $HOME|
varient to display your home directory in your sh script.

\noindent
\verb*|This is    4 space here.|

\section{\texttt{verbatim} environment}

Here is a sh script to determine if on GNU/Linux system.

\begin{verbatim}
#!/bin/sh
Date=`date '+%y%m%d'`
if [ `uname` = Linux ]
then
  Mail=/var/spool/mail/edt1023
  Target=/mnt/hd
else
  Mail=/var/mail/edt1023
  Target=/mnt/pub
fi
\end{verbatim}


\chapter{线框}

\section{直线(rule)}
% \rule[上下位置(单位)]{宽}{高}
% 在网页中显示不出来? QQQQQ
\parskip=3pt
\parindent=0pt
This is a line.       % 画高 1pt 宽 3cm 的横线。
\rule{3cm}{1pt}
\rule[1ex]{3cm}{1pt}
\rule[-1ex]{3cm}{1pt}
\rule[-1ex]{3cm}{1pt}

\rule{1pt}{3cm}      % 画高 3cm 宽 1pt 的直线。

\rule{3cm}{0pt}TEST. % 把 TEST 向右推 3cm。

\rule{2cm}{3cm}      % 画高 3cm 宽 2cm 的实体方框。

% 颜色显示不出来? QQQQQ
\textcolor{blue}{This is color lines.}
\textcolor{red}{\rule{3cm}{1pt}}        % 有颜色的线框。
\textcolor{green}{\rule[1ex]{3cm}{1pt}}
\textcolor{blue}{\rule[-1ex]{3cm}{1pt}}

\section{文字底线(underline)}
有时候我们希望在书写文字的同时,也\underline{在其下画线}。
LaTeX 有现成的指令可以使用
The tale, the Parable, and the \underline{Fable are} all common and popular


\section{方框(box)}
\parskip=3ex
\parindent=0pt
\frame{This is frame.}
\fbox{This is fbox.}
\mbox{This is mbox.} % 方框不可见

\framebox{This is a framebox with no argumant.}

\framebox[1.5\width]{This is a framebox.}

\framebox[1.5\width][l]{This is a framebox with \texttt{l}.}

\framebox[1.5\width][r]{This is a framebox with \texttt{r}.}

\framebox[1.5\width][s]{This is a framebox with \texttt{s}.}

This is baseline.
% \raisebox{上下位置(单位)}[深度][高度]{文字内容}
\raisebox{3ex}[5\height]{This is a raisebox which lift 3ex.}

This is baseline.
\fbox{\raisebox{-3ex}[5\height]{This is a raisebox which lift $-$3ex.}}

\fboxrule=1.5pt
\fboxsep=8pt
\framebox[1.5\width][s]{This is a framebox with \texttt{s}.}


\section{段落方框}
% \parbox[对齐方式][高度][内文位置]{宽度}{文字内容}

% \begin{minipage}[对齐方式][高度][内文位置]{宽度}
%   段落内容
% \end{minipage}

\parbox[t][22][3]{100}{文字内容jaljfjfjfoijoaisfj}
\parbox[b][22][3]{100}{文字内容jaljfjfjfoijoaisfj}

\begin{minipage}[c][22][3]{10}
  段落内容
  asfjljfjf
\end{minipage}


\chapter{表格的处理}
\section{tabbing 环境}
\begin{tabbing} % 在网页中排列的不是很整齐
% 这里以 10 个 x 为栏位的宽度,这里的 \kill 表示这一行是不印出来的
xxxxxxxxxx\=xxxxxxxxxx\=xxxxxxxxxx \kill
column1 \> column2 \> column3 \\
item1   \> item2   \> item3   \\
itemA   \> itemB   \> itemC
\end{tabbing}

\section{tabular 环境}
\subsection{tabular 表格的基本结构}
% 其中 [t] 表示 top,也可以是 b 表示 bottom,或 c 代表 center
% lll 是在指定各栏位内容在小方框内的置放位置,l 表示靠左(left),r 表示靠右(right),c 表示置中(center)
% \hline 是画一条横线的意思
\begin{tabular}[t]{lcr} % {|l|l|l|} 不一样的效果
\hline
column1 & column2 & column3 \\
\hline
\hline
item1   & item2   & item3 \\
itemA   & itemB   & itemC \\
\hline
\end{tabular}

\subsection{tabular 环境对栏位的调整}
\usepackage{textcomp}             % for \textcelsius
\renewcommand{\arraystretch}{1.2} % 将表格行间距加大为原来的 1.2 倍
\arrayrulewidth=1pt               % 调整线条粗细为 1pt
\tabcolsep=12pt                   % 调整栏间距为 24pt
\centering
\subsubsection*{SPECIFIC HEATS (20 \textcelsius\ AND 1 ATM)}
\begin{tabular}{@{\sf }lll@{}}    % 第一栏位使用 sans serif 字族
\hline
 & \multicolumn{2}{c}{\bf Specific Heats} \\ % 跨二三栏排版,文字置中
\cline{2-3}                                  % 只画二三栏横线
 & $c$ (J/kg$\cdot$K) & $C$ (J/mol$\cdot$K) \\
\hline
Aluminum     & 900  & 24.3 \\
Copper       & 385  & 24.4 \\
Gold         & 130  & 25.6 \\
Steel/Iron   & 450  & 25.0 \\
Lead         & 130  & 26.8 \\
Mercury      & 140  & 28.0 \\
Water        & 4190 & 75.4 \\
Ice ($-$10 \textcelsius) & 2100 & 38 \\
\hline
\end{tabular}


% p{} 指令的使用时机是某一个栏位的文字比较多,需限定栏位的宽度让他自动折行的情形
\renewcommand{\arraystretch}{1.2} % 将表格行间距加大为原来的 1.2 倍
\centering
\subsubsection*{Yi Syllables Area Character Blocks}
\begin{tabular}{@{}llp{6cm}@{}}
\hline
Start & End & Character Block Name \\
\hline
A000  & A48F  & Yi Syllables.

                Yi also known as Lolo, is a script resembling Chinese
                in overall shaps that is used in the Yunnan province
                China. \\
A490  & A4CF  & Yi Radicals.

                Basic units of the Yi syllables. \\
\hline
\end{tabular}


\section{tabularx 巨集套件}
\usepackage{tabularx}
\parindent=0pt
\renewcommand{\arraystretch}{1.2}
\centering

\subsubsection*{\texttt{tabular*} environment}

\begin{tabular*}{8cm}{llr} % 8cm(单元格长度), lll(对齐方式)
\hline
Start & End  & Character Block Name \\
\hline
3400  & 4DB5 & CJK Unified Ideographs Extension A \\
4E00  & 9FFF & CJK Unified Ideographs \\
\hline
\end{tabular*}

\subsubsection*{\textsf{tabularx} package}
\begin{tabularx}{8cm}{llX}  % 8cm 减去前两个栏位宽度后,剩下的通通给
\hline                      % 第三栏位使用,文字超出的部份会自动折行
Start & End  & Character Block Name \\
\hline
3400  & 4DB5 & CJK Unified Ideographs Extension A \\
4E00  & 9FFF & CJK Unified Ideographs \\
\hline
\end{tabularx}


\section{小数点对齐(dcolumn)}
\begin{tabular}{lllll}
\toprule
      & headA & headB & headC & headD \\
\midrule
test1 & 7.879  & 921.661 & 1382.81 & 998.98 \\
test2 & 1.97   & 35.21   & 321.3   & 4791112.11 \\
test3 & 211.97 & 5.2     & 213.629 & 748261594.106 \\
\bottomrule
\end{tabular}

% 我们只要把 tabular 的后面参数改成: ... 就可以让小数点对齐 ? QQQQQ
\usepackage{booktabs,dcolumn}
\newcolumntype{z}[1]{D{.}{.}{#1}}
\begin{tabular}{lz{3}z{3}z{3}z{3}}
\toprule
      & $headA$ & $headB$ & $headC$ & $headD$ \\
\midrule
test1 & 7.879  & 921.661 & 1382.81 & 998.98 \\
test2 & 1.97   & 35.21   & 321.3   & 4791112.11 \\
test3 & 211.97 & 5.2     & 213.629 & 748261594.106 \\
\bottomrule
\end{tabular}


\chapter{图形的处理}
\section{简化座标位置}
\usepackage{epic}
\parindent=0pt
\unitlength=1mm
\begin{picture}(80, 60)
\multiput(5, 0)(5, 0){15}{\line(0, 1){60}} % 画 15 条直线,每隔 5mm 一条
\multiput(0, 5)(0, 5){11}{\line(1, 0){80}} % 画 11 条横线,每隔 5mm 一条
\thicklines
\put(0, 0){\vector(0, 1){60}} % 画 y 轴
\put(0, 0){\vector(1, 0){80}} % 画 x 轴
\put(0, 0){\circle*{1}}       % 画圆点,实心粗点
\put(-5, -5){$O(0, 0)$}       % 标上原点的座标
\put(-5, 60){$y$}             % 标上 y 轴字样
\put(80, -5){$x$}             % 标上 x 轴字样
\end{picture}


\section{MetaPost}

\subsection{和 LaTeX 的配合}
% 编译方法:
%% 先 mptopdf test-yi.mp ; 后 pdflatex hello.tex
\vspace{10ex}
\begin{figure}[h]
\centering
\includegraphics{test-yi.1}
\caption{aaaaaa}
\end{figure}


\end{document}  % 这段放在文稿最后即可
